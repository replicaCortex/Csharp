%! TeX program = lualatex

\documentclass[12pt]{article}

\usepackage{cmap}

\usepackage[english, russian]{babel}


\usepackage{float}
\usepackage{fontspec}
\usepackage{libertinus}
\setmonofont{Source Code Pro}

\usepackage{microtype}
\usepackage{graphicx}
\usepackage{xcolor} % Цвета

\usepackage{subfig}

\usepackage{listings}

\usepackage[left=2.5cm, right=2.5cm, top=2cm, bottom=2cm]{geometry}

\usepackage[mocha, textcolor=true, pagecolor=true]{catppuccinpalette}
% \usepackage[latte, textcolor=true, pagecolor=true]{catppuccinpalette}


\definecolor{codegreen}{rgb}{0,0.6,0}
\definecolor{codegray}{rgb}{0.5,0.5,0.5}
\definecolor{codepurple}{rgb}{0.58,0,0.82}
\definecolor{codeBlue}{rgb}{0.008,0.431,0.757}

\lstdefinestyle{mystyle}{
    commentstyle=\color{codegreen},
    keywordstyle=\color{codeBlue},
    numberstyle=\tiny\color{codegray},
    stringstyle=\color{codepurple},
    basicstyle=\ttfamily\footnotesize,
    breakatwhitespace=false,         
    frame=leftline,
    xleftmargin=8pt,
    breaklines=true,                 
    captionpos=b,                    
    keepspaces=true,                 
    numbers=left,                    
    numbersep=5pt,                  
    showspaces=false,                
    showstringspaces=false,
    showtabs=false,                  
    tabsize=2
}

\lstset{style=mystyle}

\begin{document}

\section*{Описание алгоритма}

Алгоритм принимает на вход массив \( flowerbed \), где 0 обозначает пустое место, а 1 — посаженный цветок, и число \( n \) — количество цветов, которое нужно посадить. Цветы нельзя сажать рядом.

Алгоритм работает следующим образом:
\begin{enumerate}
	\item \textbf{Добавление нулей:} В начало и конец массива \( flowerbed \) добавляются нули, чтобы упростить обработку краевых случаев.
	\item \textbf{Скользящее окно:} По массиву перемещается окно фиксированного размера.
	\item \textbf{Проверка:} На каждой итерации проверяется, есть ли в текущем окне единица.
	\item \textbf{Посадка:} Если единицы нет, алгоритм уменьшает счетчик \( n \) и "сажает" цветок в середину окна (устанавливает элемент массива в 1).
	\item \textbf{Сдвиг:} Сдвигает окно на одну ячейку вправо.
	\item \textbf{Результат:} Алгоритм возвращает \( \text{True} \), если \( n \) становится равным 0, и \( \text{False} \) в противном случае.
\end{enumerate}

\subsection*{Визуализация}
\begin{figure}[H]
	\centering
	\subfloat[Добавление нулей]{%
		\includegraphics[width=0.45\textwidth]{assets/zero add.png}%
	}\hfill
	\subfloat[Окно на первых трёх элементах]{%
		\includegraphics[width=0.45\textwidth]{assets/win.png}%
	}

	\subfloat[Сдвиг окна до трёх нулей]{%
		\includegraphics[width=0.45\textwidth]{assets/movewin.png}%
	}\hfill
	\subfloat[Замена центрального нуля на 1]{%
		\includegraphics[width=0.45\textwidth]{assets/replaceZeroToOne.png}%
	}
	\caption{Этапы обработки массива с использованием скользящего окна}
\end{figure}

\break

\subsection*{Листинг}

\begin{lstlisting}[language=Python, caption=Csharp, label=code:PyTest]
public bool Flowerbed(List<int> flowerbed, int n)
{
    flowerbed.Insert(0, 0);
    flowerbed.Add(0);

    int start = 0;
    int end = 3;

    while (n > 0 && end < flowerbed.Count)
    { 
        List<int> subflower = flowerbed.GetRange(start, end - start);
        if (!subflower.Contains(1))
        {
            n--;
            flowerbed[(end + start) / 2] = 1; 
        }
        start++;
        end++;
    }

    return n == 0;
}
\end{lstlisting}

\end{document}
